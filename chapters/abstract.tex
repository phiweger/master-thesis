\section*{Abstract}

Emergent infectious diseases are a growing problem due to changes in our modern environment. Yet, as the Ebola epidemic of 2014 - 2016 in Western Africa has shown, we are ill-prepared, mainly because of two shortcomings: First, we have no vaccine for many of the (especially viral) emergent diseases such as Nipah, MERS and Lassa virus. Second, our way to exchange and curate information is very inefficient, leading to large delays between data generation and action upon that data: For example, the first comprehensive analysis of the spacio-temporal dynamics of the Ebola outbreak took over a year to complete.

Engaging the first problem, a technique called ``codon deoptimization'' is promising to generate live vaccines quickly and in theory for arbitrary viruses. We applied machine learning to help make the design of deoptimized vaccine candidates more informed and thus effective.

To encounter the second problem we designed and implemented a new data structure and exchange protocol. It serves as a proof of concept and illustrates how such structures can dramatically shorten the time from data to action, which is of vital importance in the context of public health surveillance.
